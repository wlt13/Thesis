\section{Introduction}
Epoxy is a thermoset polymer, which is highly crosslinked and brittle; hence there is a need to toughen epoxy for use in engineering applications. This project uses silica nanoparticles, core-shell rubber (CSR) particles and hybrids of both particle types, to investigate the toughening effect of epoxy with different weight \% of nanoparticles at both quasi-static and high test rates. In addition, the work will investigate the synergistic toughening effects of combining silica nanoparticles with micron-sized rubber particles. 
There are two aspects in the effect of test rate study: the effect of different weight \% of nanoparticles added, and the effect of different test rate (i.e. quasi-static and high rate). Silica particles will be added in at concentrations of 0.5, 1, 2, 3, 5, 10, 15 and 20 weight \%, and at the maximum possible concentration (25.4 weight \%). CSR particles will be added in at the same weight percentages, but up to a maximum of 10 weight \%. In previous work[1-3], the effects of different weight \% of silica have been investigated with relatively large weight percentages of nanoparticles (of 10 weight \% and above, e.g. from Hsieh et al. [2] and Mohammed et al. [4]) but not with small weight percentages. It has been suggested that small percentages of silica nanoparticles are more effective at toughening epoxy than large weight percentages, but this has not yet been investigated. Hence this study will provide more information about the effect of small percentages of silica. This effect can be shown and explained by comparing the fracture energy, $G_c$, against the weight \% of silica nanoparticles. The toughening mechanisms have been identified as shear yielding in the epoxy plus debonding of the particles followed by void growth of the epoxy. The results would be expected to show a positive increase in toughness as the percentage of nanoparticles increases, and a plateau at the maximum point. However, only about 15\% of the silica nanoparticles have been observed to debond[2]. The small weight \% of $SiO_2$ particles added in could have a greater effect on the overall curve, as a higher \% of the silica particles may be able to undergo debonding and void growth than at high \% of silica particles, and hence a steeper increase in $G_c$ vs wt\% than observed in the literature, such as from Hsieh et al. [2], at small weight\% may be expected.
For the effect of test rate, this study will start with a quasi-static rate, and then investigate the high rate effect. The higher test rate is expected to produce an increase in brittleness, and hence a reduction in the fracture energy. The hard silica particles are expected to show less of a reduction of toughness with increasing test rate when compared to the soft rubber particles. A fracture mechanics approach will be used to characterise the toughness of the epoxy. 